
% Default to the notebook output style

    


% Inherit from the specified cell style.




    
\documentclass[11pt]{article}

    
    
    \usepackage[T1]{fontenc}
    % Nicer default font (+ math font) than Computer Modern for most use cases
    \usepackage{mathpazo}

    % Basic figure setup, for now with no caption control since it's done
    % automatically by Pandoc (which extracts ![](path) syntax from Markdown).
    \usepackage{graphicx}
    % We will generate all images so they have a width \maxwidth. This means
    % that they will get their normal width if they fit onto the page, but
    % are scaled down if they would overflow the margins.
    \makeatletter
    \def\maxwidth{\ifdim\Gin@nat@width>\linewidth\linewidth
    \else\Gin@nat@width\fi}
    \makeatother
    \let\Oldincludegraphics\includegraphics
    % Set max figure width to be 80% of text width, for now hardcoded.
    \renewcommand{\includegraphics}[1]{\Oldincludegraphics[width=.8\maxwidth]{#1}}
    % Ensure that by default, figures have no caption (until we provide a
    % proper Figure object with a Caption API and a way to capture that
    % in the conversion process - todo).
    \usepackage{caption}
    \DeclareCaptionLabelFormat{nolabel}{}
    \captionsetup{labelformat=nolabel}

    \usepackage{adjustbox} % Used to constrain images to a maximum size 
    \usepackage{xcolor} % Allow colors to be defined
    \usepackage{enumerate} % Needed for markdown enumerations to work
    \usepackage{geometry} % Used to adjust the document margins
    \usepackage{amsmath} % Equations
    \usepackage{amssymb} % Equations
    \usepackage{textcomp} % defines textquotesingle
    % Hack from http://tex.stackexchange.com/a/47451/13684:
    \AtBeginDocument{%
        \def\PYZsq{\textquotesingle}% Upright quotes in Pygmentized code
    }
    \usepackage{upquote} % Upright quotes for verbatim code
    \usepackage{eurosym} % defines \euro
    \usepackage[mathletters]{ucs} % Extended unicode (utf-8) support
    \usepackage[utf8x]{inputenc} % Allow utf-8 characters in the tex document
    \usepackage{fancyvrb} % verbatim replacement that allows latex
    \usepackage{grffile} % extends the file name processing of package graphics 
                         % to support a larger range 
    % The hyperref package gives us a pdf with properly built
    % internal navigation ('pdf bookmarks' for the table of contents,
    % internal cross-reference links, web links for URLs, etc.)
    \usepackage{hyperref}
    \usepackage{longtable} % longtable support required by pandoc >1.10
    \usepackage{booktabs}  % table support for pandoc > 1.12.2
    \usepackage[inline]{enumitem} % IRkernel/repr support (it uses the enumerate* environment)
    \usepackage[normalem]{ulem} % ulem is needed to support strikethroughs (\sout)
                                % normalem makes italics be italics, not underlines
    

    
    
    % Colors for the hyperref package
    \definecolor{urlcolor}{rgb}{0,.145,.698}
    \definecolor{linkcolor}{rgb}{.71,0.21,0.01}
    \definecolor{citecolor}{rgb}{.12,.54,.11}

    % ANSI colors
    \definecolor{ansi-black}{HTML}{3E424D}
    \definecolor{ansi-black-intense}{HTML}{282C36}
    \definecolor{ansi-red}{HTML}{E75C58}
    \definecolor{ansi-red-intense}{HTML}{B22B31}
    \definecolor{ansi-green}{HTML}{00A250}
    \definecolor{ansi-green-intense}{HTML}{007427}
    \definecolor{ansi-yellow}{HTML}{DDB62B}
    \definecolor{ansi-yellow-intense}{HTML}{B27D12}
    \definecolor{ansi-blue}{HTML}{208FFB}
    \definecolor{ansi-blue-intense}{HTML}{0065CA}
    \definecolor{ansi-magenta}{HTML}{D160C4}
    \definecolor{ansi-magenta-intense}{HTML}{A03196}
    \definecolor{ansi-cyan}{HTML}{60C6C8}
    \definecolor{ansi-cyan-intense}{HTML}{258F8F}
    \definecolor{ansi-white}{HTML}{C5C1B4}
    \definecolor{ansi-white-intense}{HTML}{A1A6B2}

    % commands and environments needed by pandoc snippets
    % extracted from the output of `pandoc -s`
    \providecommand{\tightlist}{%
      \setlength{\itemsep}{0pt}\setlength{\parskip}{0pt}}
    \DefineVerbatimEnvironment{Highlighting}{Verbatim}{commandchars=\\\{\}}
    % Add ',fontsize=\small' for more characters per line
    \newenvironment{Shaded}{}{}
    \newcommand{\KeywordTok}[1]{\textcolor[rgb]{0.00,0.44,0.13}{\textbf{{#1}}}}
    \newcommand{\DataTypeTok}[1]{\textcolor[rgb]{0.56,0.13,0.00}{{#1}}}
    \newcommand{\DecValTok}[1]{\textcolor[rgb]{0.25,0.63,0.44}{{#1}}}
    \newcommand{\BaseNTok}[1]{\textcolor[rgb]{0.25,0.63,0.44}{{#1}}}
    \newcommand{\FloatTok}[1]{\textcolor[rgb]{0.25,0.63,0.44}{{#1}}}
    \newcommand{\CharTok}[1]{\textcolor[rgb]{0.25,0.44,0.63}{{#1}}}
    \newcommand{\StringTok}[1]{\textcolor[rgb]{0.25,0.44,0.63}{{#1}}}
    \newcommand{\CommentTok}[1]{\textcolor[rgb]{0.38,0.63,0.69}{\textit{{#1}}}}
    \newcommand{\OtherTok}[1]{\textcolor[rgb]{0.00,0.44,0.13}{{#1}}}
    \newcommand{\AlertTok}[1]{\textcolor[rgb]{1.00,0.00,0.00}{\textbf{{#1}}}}
    \newcommand{\FunctionTok}[1]{\textcolor[rgb]{0.02,0.16,0.49}{{#1}}}
    \newcommand{\RegionMarkerTok}[1]{{#1}}
    \newcommand{\ErrorTok}[1]{\textcolor[rgb]{1.00,0.00,0.00}{\textbf{{#1}}}}
    \newcommand{\NormalTok}[1]{{#1}}
    
    % Additional commands for more recent versions of Pandoc
    \newcommand{\ConstantTok}[1]{\textcolor[rgb]{0.53,0.00,0.00}{{#1}}}
    \newcommand{\SpecialCharTok}[1]{\textcolor[rgb]{0.25,0.44,0.63}{{#1}}}
    \newcommand{\VerbatimStringTok}[1]{\textcolor[rgb]{0.25,0.44,0.63}{{#1}}}
    \newcommand{\SpecialStringTok}[1]{\textcolor[rgb]{0.73,0.40,0.53}{{#1}}}
    \newcommand{\ImportTok}[1]{{#1}}
    \newcommand{\DocumentationTok}[1]{\textcolor[rgb]{0.73,0.13,0.13}{\textit{{#1}}}}
    \newcommand{\AnnotationTok}[1]{\textcolor[rgb]{0.38,0.63,0.69}{\textbf{\textit{{#1}}}}}
    \newcommand{\CommentVarTok}[1]{\textcolor[rgb]{0.38,0.63,0.69}{\textbf{\textit{{#1}}}}}
    \newcommand{\VariableTok}[1]{\textcolor[rgb]{0.10,0.09,0.49}{{#1}}}
    \newcommand{\ControlFlowTok}[1]{\textcolor[rgb]{0.00,0.44,0.13}{\textbf{{#1}}}}
    \newcommand{\OperatorTok}[1]{\textcolor[rgb]{0.40,0.40,0.40}{{#1}}}
    \newcommand{\BuiltInTok}[1]{{#1}}
    \newcommand{\ExtensionTok}[1]{{#1}}
    \newcommand{\PreprocessorTok}[1]{\textcolor[rgb]{0.74,0.48,0.00}{{#1}}}
    \newcommand{\AttributeTok}[1]{\textcolor[rgb]{0.49,0.56,0.16}{{#1}}}
    \newcommand{\InformationTok}[1]{\textcolor[rgb]{0.38,0.63,0.69}{\textbf{\textit{{#1}}}}}
    \newcommand{\WarningTok}[1]{\textcolor[rgb]{0.38,0.63,0.69}{\textbf{\textit{{#1}}}}}
    
    
    % Define a nice break command that doesn't care if a line doesn't already
    % exist.
    \def\br{\hspace*{\fill} \\* }
    % Math Jax compatability definitions
    \def\gt{>}
    \def\lt{<}
    % Document parameters
    \title{Calculus\_A4}
    
    
    

    % Pygments definitions
    
\makeatletter
\def\PY@reset{\let\PY@it=\relax \let\PY@bf=\relax%
    \let\PY@ul=\relax \let\PY@tc=\relax%
    \let\PY@bc=\relax \let\PY@ff=\relax}
\def\PY@tok#1{\csname PY@tok@#1\endcsname}
\def\PY@toks#1+{\ifx\relax#1\empty\else%
    \PY@tok{#1}\expandafter\PY@toks\fi}
\def\PY@do#1{\PY@bc{\PY@tc{\PY@ul{%
    \PY@it{\PY@bf{\PY@ff{#1}}}}}}}
\def\PY#1#2{\PY@reset\PY@toks#1+\relax+\PY@do{#2}}

\expandafter\def\csname PY@tok@gd\endcsname{\def\PY@tc##1{\textcolor[rgb]{0.63,0.00,0.00}{##1}}}
\expandafter\def\csname PY@tok@gu\endcsname{\let\PY@bf=\textbf\def\PY@tc##1{\textcolor[rgb]{0.50,0.00,0.50}{##1}}}
\expandafter\def\csname PY@tok@gt\endcsname{\def\PY@tc##1{\textcolor[rgb]{0.00,0.27,0.87}{##1}}}
\expandafter\def\csname PY@tok@gs\endcsname{\let\PY@bf=\textbf}
\expandafter\def\csname PY@tok@gr\endcsname{\def\PY@tc##1{\textcolor[rgb]{1.00,0.00,0.00}{##1}}}
\expandafter\def\csname PY@tok@cm\endcsname{\let\PY@it=\textit\def\PY@tc##1{\textcolor[rgb]{0.25,0.50,0.50}{##1}}}
\expandafter\def\csname PY@tok@vg\endcsname{\def\PY@tc##1{\textcolor[rgb]{0.10,0.09,0.49}{##1}}}
\expandafter\def\csname PY@tok@vi\endcsname{\def\PY@tc##1{\textcolor[rgb]{0.10,0.09,0.49}{##1}}}
\expandafter\def\csname PY@tok@vm\endcsname{\def\PY@tc##1{\textcolor[rgb]{0.10,0.09,0.49}{##1}}}
\expandafter\def\csname PY@tok@mh\endcsname{\def\PY@tc##1{\textcolor[rgb]{0.40,0.40,0.40}{##1}}}
\expandafter\def\csname PY@tok@cs\endcsname{\let\PY@it=\textit\def\PY@tc##1{\textcolor[rgb]{0.25,0.50,0.50}{##1}}}
\expandafter\def\csname PY@tok@ge\endcsname{\let\PY@it=\textit}
\expandafter\def\csname PY@tok@vc\endcsname{\def\PY@tc##1{\textcolor[rgb]{0.10,0.09,0.49}{##1}}}
\expandafter\def\csname PY@tok@il\endcsname{\def\PY@tc##1{\textcolor[rgb]{0.40,0.40,0.40}{##1}}}
\expandafter\def\csname PY@tok@go\endcsname{\def\PY@tc##1{\textcolor[rgb]{0.53,0.53,0.53}{##1}}}
\expandafter\def\csname PY@tok@cp\endcsname{\def\PY@tc##1{\textcolor[rgb]{0.74,0.48,0.00}{##1}}}
\expandafter\def\csname PY@tok@gi\endcsname{\def\PY@tc##1{\textcolor[rgb]{0.00,0.63,0.00}{##1}}}
\expandafter\def\csname PY@tok@gh\endcsname{\let\PY@bf=\textbf\def\PY@tc##1{\textcolor[rgb]{0.00,0.00,0.50}{##1}}}
\expandafter\def\csname PY@tok@ni\endcsname{\let\PY@bf=\textbf\def\PY@tc##1{\textcolor[rgb]{0.60,0.60,0.60}{##1}}}
\expandafter\def\csname PY@tok@nl\endcsname{\def\PY@tc##1{\textcolor[rgb]{0.63,0.63,0.00}{##1}}}
\expandafter\def\csname PY@tok@nn\endcsname{\let\PY@bf=\textbf\def\PY@tc##1{\textcolor[rgb]{0.00,0.00,1.00}{##1}}}
\expandafter\def\csname PY@tok@no\endcsname{\def\PY@tc##1{\textcolor[rgb]{0.53,0.00,0.00}{##1}}}
\expandafter\def\csname PY@tok@na\endcsname{\def\PY@tc##1{\textcolor[rgb]{0.49,0.56,0.16}{##1}}}
\expandafter\def\csname PY@tok@nb\endcsname{\def\PY@tc##1{\textcolor[rgb]{0.00,0.50,0.00}{##1}}}
\expandafter\def\csname PY@tok@nc\endcsname{\let\PY@bf=\textbf\def\PY@tc##1{\textcolor[rgb]{0.00,0.00,1.00}{##1}}}
\expandafter\def\csname PY@tok@nd\endcsname{\def\PY@tc##1{\textcolor[rgb]{0.67,0.13,1.00}{##1}}}
\expandafter\def\csname PY@tok@ne\endcsname{\let\PY@bf=\textbf\def\PY@tc##1{\textcolor[rgb]{0.82,0.25,0.23}{##1}}}
\expandafter\def\csname PY@tok@nf\endcsname{\def\PY@tc##1{\textcolor[rgb]{0.00,0.00,1.00}{##1}}}
\expandafter\def\csname PY@tok@si\endcsname{\let\PY@bf=\textbf\def\PY@tc##1{\textcolor[rgb]{0.73,0.40,0.53}{##1}}}
\expandafter\def\csname PY@tok@s2\endcsname{\def\PY@tc##1{\textcolor[rgb]{0.73,0.13,0.13}{##1}}}
\expandafter\def\csname PY@tok@nt\endcsname{\let\PY@bf=\textbf\def\PY@tc##1{\textcolor[rgb]{0.00,0.50,0.00}{##1}}}
\expandafter\def\csname PY@tok@nv\endcsname{\def\PY@tc##1{\textcolor[rgb]{0.10,0.09,0.49}{##1}}}
\expandafter\def\csname PY@tok@s1\endcsname{\def\PY@tc##1{\textcolor[rgb]{0.73,0.13,0.13}{##1}}}
\expandafter\def\csname PY@tok@dl\endcsname{\def\PY@tc##1{\textcolor[rgb]{0.73,0.13,0.13}{##1}}}
\expandafter\def\csname PY@tok@ch\endcsname{\let\PY@it=\textit\def\PY@tc##1{\textcolor[rgb]{0.25,0.50,0.50}{##1}}}
\expandafter\def\csname PY@tok@m\endcsname{\def\PY@tc##1{\textcolor[rgb]{0.40,0.40,0.40}{##1}}}
\expandafter\def\csname PY@tok@gp\endcsname{\let\PY@bf=\textbf\def\PY@tc##1{\textcolor[rgb]{0.00,0.00,0.50}{##1}}}
\expandafter\def\csname PY@tok@sh\endcsname{\def\PY@tc##1{\textcolor[rgb]{0.73,0.13,0.13}{##1}}}
\expandafter\def\csname PY@tok@ow\endcsname{\let\PY@bf=\textbf\def\PY@tc##1{\textcolor[rgb]{0.67,0.13,1.00}{##1}}}
\expandafter\def\csname PY@tok@sx\endcsname{\def\PY@tc##1{\textcolor[rgb]{0.00,0.50,0.00}{##1}}}
\expandafter\def\csname PY@tok@bp\endcsname{\def\PY@tc##1{\textcolor[rgb]{0.00,0.50,0.00}{##1}}}
\expandafter\def\csname PY@tok@c1\endcsname{\let\PY@it=\textit\def\PY@tc##1{\textcolor[rgb]{0.25,0.50,0.50}{##1}}}
\expandafter\def\csname PY@tok@fm\endcsname{\def\PY@tc##1{\textcolor[rgb]{0.00,0.00,1.00}{##1}}}
\expandafter\def\csname PY@tok@o\endcsname{\def\PY@tc##1{\textcolor[rgb]{0.40,0.40,0.40}{##1}}}
\expandafter\def\csname PY@tok@kc\endcsname{\let\PY@bf=\textbf\def\PY@tc##1{\textcolor[rgb]{0.00,0.50,0.00}{##1}}}
\expandafter\def\csname PY@tok@c\endcsname{\let\PY@it=\textit\def\PY@tc##1{\textcolor[rgb]{0.25,0.50,0.50}{##1}}}
\expandafter\def\csname PY@tok@mf\endcsname{\def\PY@tc##1{\textcolor[rgb]{0.40,0.40,0.40}{##1}}}
\expandafter\def\csname PY@tok@err\endcsname{\def\PY@bc##1{\setlength{\fboxsep}{0pt}\fcolorbox[rgb]{1.00,0.00,0.00}{1,1,1}{\strut ##1}}}
\expandafter\def\csname PY@tok@mb\endcsname{\def\PY@tc##1{\textcolor[rgb]{0.40,0.40,0.40}{##1}}}
\expandafter\def\csname PY@tok@ss\endcsname{\def\PY@tc##1{\textcolor[rgb]{0.10,0.09,0.49}{##1}}}
\expandafter\def\csname PY@tok@sr\endcsname{\def\PY@tc##1{\textcolor[rgb]{0.73,0.40,0.53}{##1}}}
\expandafter\def\csname PY@tok@mo\endcsname{\def\PY@tc##1{\textcolor[rgb]{0.40,0.40,0.40}{##1}}}
\expandafter\def\csname PY@tok@kd\endcsname{\let\PY@bf=\textbf\def\PY@tc##1{\textcolor[rgb]{0.00,0.50,0.00}{##1}}}
\expandafter\def\csname PY@tok@mi\endcsname{\def\PY@tc##1{\textcolor[rgb]{0.40,0.40,0.40}{##1}}}
\expandafter\def\csname PY@tok@kn\endcsname{\let\PY@bf=\textbf\def\PY@tc##1{\textcolor[rgb]{0.00,0.50,0.00}{##1}}}
\expandafter\def\csname PY@tok@cpf\endcsname{\let\PY@it=\textit\def\PY@tc##1{\textcolor[rgb]{0.25,0.50,0.50}{##1}}}
\expandafter\def\csname PY@tok@kr\endcsname{\let\PY@bf=\textbf\def\PY@tc##1{\textcolor[rgb]{0.00,0.50,0.00}{##1}}}
\expandafter\def\csname PY@tok@s\endcsname{\def\PY@tc##1{\textcolor[rgb]{0.73,0.13,0.13}{##1}}}
\expandafter\def\csname PY@tok@kp\endcsname{\def\PY@tc##1{\textcolor[rgb]{0.00,0.50,0.00}{##1}}}
\expandafter\def\csname PY@tok@w\endcsname{\def\PY@tc##1{\textcolor[rgb]{0.73,0.73,0.73}{##1}}}
\expandafter\def\csname PY@tok@kt\endcsname{\def\PY@tc##1{\textcolor[rgb]{0.69,0.00,0.25}{##1}}}
\expandafter\def\csname PY@tok@sc\endcsname{\def\PY@tc##1{\textcolor[rgb]{0.73,0.13,0.13}{##1}}}
\expandafter\def\csname PY@tok@sb\endcsname{\def\PY@tc##1{\textcolor[rgb]{0.73,0.13,0.13}{##1}}}
\expandafter\def\csname PY@tok@sa\endcsname{\def\PY@tc##1{\textcolor[rgb]{0.73,0.13,0.13}{##1}}}
\expandafter\def\csname PY@tok@k\endcsname{\let\PY@bf=\textbf\def\PY@tc##1{\textcolor[rgb]{0.00,0.50,0.00}{##1}}}
\expandafter\def\csname PY@tok@se\endcsname{\let\PY@bf=\textbf\def\PY@tc##1{\textcolor[rgb]{0.73,0.40,0.13}{##1}}}
\expandafter\def\csname PY@tok@sd\endcsname{\let\PY@it=\textit\def\PY@tc##1{\textcolor[rgb]{0.73,0.13,0.13}{##1}}}

\def\PYZbs{\char`\\}
\def\PYZus{\char`\_}
\def\PYZob{\char`\{}
\def\PYZcb{\char`\}}
\def\PYZca{\char`\^}
\def\PYZam{\char`\&}
\def\PYZlt{\char`\<}
\def\PYZgt{\char`\>}
\def\PYZsh{\char`\#}
\def\PYZpc{\char`\%}
\def\PYZdl{\char`\$}
\def\PYZhy{\char`\-}
\def\PYZsq{\char`\'}
\def\PYZdq{\char`\"}
\def\PYZti{\char`\~}
% for compatibility with earlier versions
\def\PYZat{@}
\def\PYZlb{[}
\def\PYZrb{]}
\makeatother


    % Exact colors from NB
    \definecolor{incolor}{rgb}{0.0, 0.0, 0.5}
    \definecolor{outcolor}{rgb}{0.545, 0.0, 0.0}



    
    % Prevent overflowing lines due to hard-to-break entities
    \sloppy 
    % Setup hyperref package
    \hypersetup{
      breaklinks=true,  % so long urls are correctly broken across lines
      colorlinks=true,
      urlcolor=urlcolor,
      linkcolor=linkcolor,
      citecolor=citecolor,
      }
    % Slightly bigger margins than the latex defaults
    
    \geometry{verbose,tmargin=1in,bmargin=1in,lmargin=1in,rmargin=1in}
    
    

    \begin{document}
    
    
    \maketitle
    
    

    
    \section{Michelle}\label{michelle}

\subsection{Assignment 4}\label{assignment-4}

\subsection{CS111A Fall 2018}\label{cs111a-fall-2018}

    \subsection{Part A}\label{part-a}

\subsubsection{A.1}\label{a.1}

In the late 1860s, Adolf Fick, a professor of physiology in the Faculty
of Medicine in W ̈urzberg, Germany, developed one of the methods we use
today for measuring how much blood your heart pumps in a minute. Your
cardiac output as you read this sentence is probably about 7 L/min. At
rest it is likely to be a bit under 6 L/min. If you are a trained
marathon runner running a marathon, your cardiac output can be as high
as 30 L/min. Your cardiac output can be calculated with the formula

\[y = \frac{Q}{D}\]

where \(Q\) is the number of milliliters of CO2 you exhale in a minute
and \(D\) is the difference between the CO2 concentration (ml/L) in the
blood pumped to the lungs and the CO2 concentration in the blood
returning from the lungs. With \(Q = 233 ml/min\) and
\(D = 97 − 56 = 41 ml/L\), \(y = 233 ml/min/ 41 ml/L ≈ 5.68 L/min\),
fairly close to the 6 L/min that most people have at basal (resting)
conditions. (Data courtesy of J. Kenneth Herd, M.D., Quillan College of
Medicine.) Suppose that when \(Q = 233\) and \(D = 41\), we also know
that \(D\) is decreasing at the rate of 2 units a minute but that \(Q\)
remains unchanged. What is happening to the cardiac output?

    Objective function:\\
\[y = \frac{Q}{D}\]

Other information:\\
When \(Q = 233\) and \(D = 41\)\\
\(\frac{\delta D}{\delta T} = 2\)\\
\(\frac{\delta Q}{\delta T} = 0\)

What's happening to y?

\(\frac{\delta Y}{\delta T} = \frac{\frac{\delta Q}{\delta T}*D - Q*\frac{\delta D}{}}{D^2}\)
\#quotient rule

\(\frac{\delta Y}{\delta T} = (0*41 - 233*2)/41^2\)
\[\frac{\delta Y}{\delta T} = -466/1681\]

    \subsubsection{A.2}\label{a.2}

Suppose that a scientist has reason to believe that two quantities \(x\)
and \(y\) are related linearly, that is \(y = mx + b\), at least
approximately, for some values of \(m\) and \(b\). The scientist
performs an experiment and collects data in the form of points
\((x_1, y_1)\), \((x_2, y_2)\), ... ,\((x_n, y_n)\), and then plots
these points. The points don't lie exactly on a straight line, so the
scientist wants to find constants \(m\) and \(b\) so that the line
\(y = mx + b\) ``fits'' the points as well as possible.

Let \(d_i = y_i − (mx_i + b)\) be the vertical deviation of the point
\((x_i, y_i)\) from the line. The method of least squares determines
\(m\) and \(b\) so as to minimize \(\sum_{i=1}^{n} d^2_i\), the sum of
the squares of these vertical deviations. Show that, according to this
method,the line of best fit is obtained when m and b satisfy the
following system of equations:

\[m\sum_{i=1}^{n} x_i + bn = \sum_{i=1}^{n}y_i\]
\[m\sum_{i=1}^{n} x^2_i + b\sum_{i=1}^{n}x_i = \sum_{i=1}^{n}x_iy_i\]

    Line of best fit is when sum of squared errors
(\(\sum_{i=1}^{n} d^2_i\)) is minimized

\emph{Let} \(f(x) = \sum_{i=1}^{n} d^2_i\)

\emph{substitute in the formula for} \(d_i\)\\
\(f(x) = \sum_{i=1}^{n} ( (y_i − (mx_i + b))^2)\)\\
\emph{simplify} \(f(x) = \sum_{i=1}^{n} (y_i − mx_i - b)^2\)

    The sum of squared errors is minimized at a critical point within the
graph (when the instantaneous rate of change of the objective function
along any variable is 0).

Objective function:\\
\[f(x) = \sum_{i=1}^{n} (y_i − mx_i - b)^2\]

Variables: \[m\] \[b\]

The instantaneous rate of change of the objective function along any
variable is 0 when the partial derivatives are equal to 0.

\emph{partial derivative by m:}\\
\(\frac{\delta f}{\delta m} =\sum_{i=1}^{n} 2(y_i − mx_i - b)*-x_i\)
\#chain rule\\
\(\frac{\delta f}{\delta m} =-2\sum_{i=1}^{n}(y_i − mx_i - b)*x_i\)
\#rearrange\\
\emph{critical point when derivative = 0}\\
\(0 =-2\sum_{i=1}^{n}(y_i − mx_i - b)*x_i\)\\
\(0 =\sum_{i=1}^{n}(y_i − mx_i - b)*x_i\)\\
\(0 =\sum_{i=1}^{n}(x_iy_i − mx_i^2 - bx_i)\) \#expand\\
\(m\sum_{i=1}^{n}x_i^2 + b\sum_{i=1}^{n}x_i = \sum_{i=1}^{n}x_iy_i\)
\#expand and rearrange

\emph{partial derivative by b:}\\
\(\frac{\delta f}{\delta b} = \sum_{i=1}^{n} 2(y_i − mx_i - b)*-1\)
\#chain rule\\
\(\frac{\delta f}{\delta b} = -2\sum_{i=1}^{n}(y_i − mx_i - b)\)
\#rearrange\\
\emph{critical point when derivative = 0:}\\
\(0 = -2\sum_{i=1}^{n}(y_i − mx_i - b)\)\\
\(0 = \sum_{i=1}^{n}(y_i − mx_i - b)\)\\
\(0 = \sum_{i=1}^{n}y_i − m\sum_{i=1}^{n}x_i - \sum_{i=1}^{n}b)\)
\#expand\\
\(m\sum_{i=1}^{n}x_i + bn = \sum_{i=1}^{n}y_i\) \#rearrange

The partial derivatives of the objective function is 0 when \(m\) and
\(b\) satisfy the following equations:\\
\[m\sum_{i=1}^{n}x_i^2 + b\sum_{i=1}^{n}x_i = \sum_{i=1}^{n}x_iy_i\]
\[m\sum_{i=1}^{n}x_i + bn = \sum_{i=1}^{n}y_i\]

    \textbf{Testing if it's min}

\(\frac{\delta f}{\delta m} =-2\sum_{i=1}^{n}(y_i − mx_i - b)*x_i\)\\
\(\frac{\delta f}{\delta m} =-2\sum_{i=1}^{n}(y_ix_i − mx_i^2 - bx_i)\)
\#expand\\
\(\frac{\delta f^2}{\delta m^2} =4\sum_{i=1}^{n}x_i\)

\(\frac{\delta f}{\delta b} = -2\sum_{i=1}^{n}(y_i − mx_i - b)\)\\
\(\frac{\delta f}{\delta b} = -2(\sum_{i=1}^{n}y_i − \sum_{i=1}^{n}mx_i - nb)\)\\
\(\frac{\delta f^2}{\delta b^2} = 2n\)

\emph{We can assume that n will be positive (number of datapoints
\textgreater{} 0) but we cannot assume that the sum of all x values will
be positive. The sum of x values will not be testable since it will
depend on the dataset.}

\emph{Algebraic proof will be far beyond the scope of this course}

\emph{That said, since the function \(f(x) = \sum_{i=1}^{n} d^2_i\) is a
squared quadratic function, the equation will be a concave up function.
Hence, the only critical point should be a min point. Thus we can safely
assume that this is a min point.}

    \subsection{Part B}\label{part-b}

\subsubsection{B.1}\label{b.1}

Airlines want to maximize profits with every flight. Since cargo is
expensive to fly, airlines are now charging baggage fees. With baggage
fees, customers are less likely to take big bags with them, giving less
of a profit to the airline. Let's model a specific example:

\begin{enumerate}
\def\labelenumi{(\alph{enumi})}
\tightlist
\item
  Airlines noticed that if they charge \$20 for each bag, then in a
  given flight, 50 bags will be checked. They also noticed that for
  every \$2 increase in the baggage fee, 2 fewer bags will be checked
  in. How can we model the revenue due to checked bags as a function of
  the number of price increases x? What are the constraints on the value
  of x?
\end{enumerate}

    \textbf{Revenue due to checked bags}\\
Objective function:\\
\(revenue = number.of.bags * (20+2x)\)\\
\emph{number of bags is a dependent variable on price}\\
\(r(p) = B(x)*(20+2x)\)

\(B(x) = a - x\) \#where a is an unknown\\
\emph{sub in the variable values we were given to find unknown: 50 bags
when \$20 charged}\\
\(50 = a -20\)\\
\(a = 70\)\\
\emph{sub the value of a back into function}\\
\(B(p) = 70 - x\)

Sub the function for number of bags back into the overall objective
function\\
\(r(p) = (70 - x)*(20+2x)\)

\textbf{What are the constraints?}\\
There will likely be a constraint on the maximal number of bags that can
be fit in the plane (max value of \(B(p)\)).

    \begin{enumerate}
\def\labelenumi{(\alph{enumi})}
\setcounter{enumi}{1}
\tightlist
\item
  There are also some costs associated with carrying bags. This specific
  company noticed that the cost of checking N bags is equal to\\
  \[C(N) = \frac{N^2}{20} + 7N\]
\end{enumerate}

Write a function that describes the profit from checked bags in terms of
the number of price increases x.

    \(r(p) = (70 - x)*(20+2x) - (\frac{N^2}{20} + 7N)\)

\emph{replace N with the function for number of bags}

\(r(p) = (70 - x)*(20+2x) - (\frac{(70-x)^2}{20} + 7(70-x))\)\\
\(r(p) = (70 - x)*(20+2x) - 7(70-x) - \frac{(70-x)^2}{20}\)

    \begin{enumerate}
\def\labelenumi{(\alph{enumi})}
\setcounter{enumi}{2}
\tightlist
\item
  Find the number of price increases x that will maximize this company's
  profits. What will be the optimal baggage fee? How many bags will be
  taken on the flight?
\end{enumerate}

    \textbf{Find Maximal Point}\\
\(r(p) = (70 - x)*(20+2x) - 7(70-x) - \frac{(70-x)^2}{20}\)

\textbf{Find critical points (instantaneous rate of change = 0)}

\(\frac {\delta r}{\delta x}\)\\
\(= \frac {\delta}{\delta x}[(70-x)(2x+20)] - 7(\frac {\delta}{\delta x}[70-x] - \frac{1}{20}\frac {\delta}{\delta x}[(70-x)^2])\)\\
\(= -1(2x+20)+2(70-x)-7(-1)-\frac{1}{20}2(70-x)(-1)\)\\
\(= -1(2x+20)+2(70-x)+7-\frac{x-70}{10}\)\\
\(= -2x-20+140-2x+7-\frac{x-70}{10}\)\\
\(= -4x+127-\frac{x-70}{10}\)\\
\(= -\frac{41x-1340}{10}\)

\emph{equate to 0}\\
\(0 = -\frac{41x-1340}{10}\)\\
\(0 = 41x-1340\)\\
\(41x=1340\)\\
\(x=\frac{1340}{41}\)

\textbf{Check that it is a maximal point}
\(\frac {\delta r}{\delta x}= -\frac{41x-1340}{10}\)\\
\(\frac {\delta r}{\delta x}= -\frac{1}{10}41x-1340\)

\(\frac {\delta r^2}{\delta x^2} = -\frac{1}{10}\frac{\delta}{\delta x}[41x-1340]\)\\
\(\frac {\delta r^2}{\delta x^2} = -\frac{1}{10}41\)\\
\(\frac {\delta r^2}{\delta x^2} = -\frac{41}{10}\)
\({\delta r^2}{\delta x^2} < 0\)\\
\emph{:. Hence this is a maximal point}

\textbf{Optimal Baggage Fee}\\
\(=\frac{1340}{41}\)

\textbf{How many bags will be on this flight?}\\
\(B(p) = 70 - (\frac{1340}{41})\)\\
\(B(p) = 70 - 37.32\)\\
\emph{round down}\\
\(= 37\)

    \subsubsection{B.2}\label{b.2}

Three alleles (alternative versions of a gene) A,B, and O determine the
four blood types A (AA or AO), B (BB or BO), O (OO), and AB. The
Hardy-Weinberg Law states that the proportion of individuals in a
population who carry two different allelles is

\[P = 2pq + 2pr + 2rq\]

where p, q, and r are the proportions of A, B, and O in the population.
Use the fact that \(p + q + r = 1\) to determine the maximal value P can
take.

    \textbf{Objective Function O(p,q,r)}\\
\(P(p,q,r) = 2pq + 2pr + 2rq\)

\textbf{Constraint Function C(p,q,r)}\\
\(p+q+r=1\)

\textbf{Find Maximum}\\
\emph{use Lagrange Multipliers}

\textbf{Find Critical Points}

\emph{find derivatives of all 3 factors in both objective and constraint
function{[}s{]}}\\
\(\frac{\delta P}{\delta p} = 2q + 2r\)\\
\(\frac{\delta P}{\delta q} = 2p + 2r\)\\
\(\frac{\delta P}{\delta r} = 2p + 2q\)

\(\frac{\delta C}{\delta p} =1\)\\
\(\frac{\delta C}{\delta q} =1\)\\
\(\frac{\delta C}{\delta r} =1\)

\emph{find equation set}\\
A: \(2q + 2r = \lambda 1\)\\
B: \(2p + 2r = \lambda 1\)\\
C: \(2p + 2q = \lambda 1\)\\
D: \(p+q+r=1\)

\emph{find variable value combinations that satisfy the equation set}\\
A-B:\\
\(2q-2p = 0\)\\
\(2q = 2p\)

B-C:\\
\(2r -2q = 0\)\\
\(2r= 2q = 2p\)

C-A:\\
\(2p -2r = 0\)\\
\(2p = 2r = 2q = 2p\)\\
E: \(p = r = q = p\)

\emph{sub into D}:\\
\(p+p+p=1\)\\
\(p=\frac{1}{3}\)

\emph{sub into E}\\
\(p = r = q = p = \frac{1}{3}\)

\emph{sub values back into objective function}\\
\(P(\frac{1}{3},\frac{1}{3},\frac{1}{3}) = 2\frac{1}{3}^2 + 2\frac{1}{3}^2 + 2\frac{1}{3}^2\)\\
\(P(\frac{1}{3},\frac{1}{3},\frac{1}{3}) = \frac{6}{9}\)

\emph{check any other point in the graph}\\
Constraint: \(p+q+r=1\)\\
\emph{a random set of variable values that satisfy the constraint}\\
\(p = 1; q = 0; r = 0\)

\(P(1,0,0) = 2(1)(0) + 2(1)(0) + 2(0)(0)\)\\
\(P(1,0,0) = 0\)\\
\(P(1,0,0) < P(\frac{1}{3},\frac{1}{3},\frac{1}{3})\)\\
\emph{Since Lagrange Multipliers method ensures you get either the max
or min point and there exists a point in the objective function that is
lower than the point that the Lagrange method gave us, we can conclude
that the point that Lagrange method gave us is not min, and therefore
must be max.}

    \section{Appendix}\label{appendix}

\textbf{\#deduction: } In situations where algebraic proof would have
been unnecessarily complex, I used deduction on mathematical laws to
prove min and max points. Most notably in A2 and B2.

\textbf{\#testability: } In question A2, I identified when specific
lines of proof would not be testable (by nature of the question or the
material covered in the course) and identified an alternative way to
prove the question.


    % Add a bibliography block to the postdoc
    
    
    
    \end{document}
