
% Default to the notebook output style

    


% Inherit from the specified cell style.




    
\documentclass[11pt]{article}

    
    
    \usepackage[T1]{fontenc}
    % Nicer default font (+ math font) than Computer Modern for most use cases
    \usepackage{mathpazo}

    % Basic figure setup, for now with no caption control since it's done
    % automatically by Pandoc (which extracts ![](path) syntax from Markdown).
    \usepackage{graphicx}
    % We will generate all images so they have a width \maxwidth. This means
    % that they will get their normal width if they fit onto the page, but
    % are scaled down if they would overflow the margins.
    \makeatletter
    \def\maxwidth{\ifdim\Gin@nat@width>\linewidth\linewidth
    \else\Gin@nat@width\fi}
    \makeatother
    \let\Oldincludegraphics\includegraphics
    % Set max figure width to be 80% of text width, for now hardcoded.
    \renewcommand{\includegraphics}[1]{\Oldincludegraphics[width=.8\maxwidth]{#1}}
    % Ensure that by default, figures have no caption (until we provide a
    % proper Figure object with a Caption API and a way to capture that
    % in the conversion process - todo).
    \usepackage{caption}
    \DeclareCaptionLabelFormat{nolabel}{}
    \captionsetup{labelformat=nolabel}

    \usepackage{adjustbox} % Used to constrain images to a maximum size 
    \usepackage{xcolor} % Allow colors to be defined
    \usepackage{enumerate} % Needed for markdown enumerations to work
    \usepackage{geometry} % Used to adjust the document margins
    \usepackage{amsmath} % Equations
    \usepackage{amssymb} % Equations
    \usepackage{textcomp} % defines textquotesingle
    % Hack from http://tex.stackexchange.com/a/47451/13684:
    \AtBeginDocument{%
        \def\PYZsq{\textquotesingle}% Upright quotes in Pygmentized code
    }
    \usepackage{upquote} % Upright quotes for verbatim code
    \usepackage{eurosym} % defines \euro
    \usepackage[mathletters]{ucs} % Extended unicode (utf-8) support
    \usepackage[utf8x]{inputenc} % Allow utf-8 characters in the tex document
    \usepackage{fancyvrb} % verbatim replacement that allows latex
    \usepackage{grffile} % extends the file name processing of package graphics 
                         % to support a larger range 
    % The hyperref package gives us a pdf with properly built
    % internal navigation ('pdf bookmarks' for the table of contents,
    % internal cross-reference links, web links for URLs, etc.)
    \usepackage{hyperref}
    \usepackage{longtable} % longtable support required by pandoc >1.10
    \usepackage{booktabs}  % table support for pandoc > 1.12.2
    \usepackage[inline]{enumitem} % IRkernel/repr support (it uses the enumerate* environment)
    \usepackage[normalem]{ulem} % ulem is needed to support strikethroughs (\sout)
                                % normalem makes italics be italics, not underlines
    

    
    
    % Colors for the hyperref package
    \definecolor{urlcolor}{rgb}{0,.145,.698}
    \definecolor{linkcolor}{rgb}{.71,0.21,0.01}
    \definecolor{citecolor}{rgb}{.12,.54,.11}

    % ANSI colors
    \definecolor{ansi-black}{HTML}{3E424D}
    \definecolor{ansi-black-intense}{HTML}{282C36}
    \definecolor{ansi-red}{HTML}{E75C58}
    \definecolor{ansi-red-intense}{HTML}{B22B31}
    \definecolor{ansi-green}{HTML}{00A250}
    \definecolor{ansi-green-intense}{HTML}{007427}
    \definecolor{ansi-yellow}{HTML}{DDB62B}
    \definecolor{ansi-yellow-intense}{HTML}{B27D12}
    \definecolor{ansi-blue}{HTML}{208FFB}
    \definecolor{ansi-blue-intense}{HTML}{0065CA}
    \definecolor{ansi-magenta}{HTML}{D160C4}
    \definecolor{ansi-magenta-intense}{HTML}{A03196}
    \definecolor{ansi-cyan}{HTML}{60C6C8}
    \definecolor{ansi-cyan-intense}{HTML}{258F8F}
    \definecolor{ansi-white}{HTML}{C5C1B4}
    \definecolor{ansi-white-intense}{HTML}{A1A6B2}

    % commands and environments needed by pandoc snippets
    % extracted from the output of `pandoc -s`
    \providecommand{\tightlist}{%
      \setlength{\itemsep}{0pt}\setlength{\parskip}{0pt}}
    \DefineVerbatimEnvironment{Highlighting}{Verbatim}{commandchars=\\\{\}}
    % Add ',fontsize=\small' for more characters per line
    \newenvironment{Shaded}{}{}
    \newcommand{\KeywordTok}[1]{\textcolor[rgb]{0.00,0.44,0.13}{\textbf{{#1}}}}
    \newcommand{\DataTypeTok}[1]{\textcolor[rgb]{0.56,0.13,0.00}{{#1}}}
    \newcommand{\DecValTok}[1]{\textcolor[rgb]{0.25,0.63,0.44}{{#1}}}
    \newcommand{\BaseNTok}[1]{\textcolor[rgb]{0.25,0.63,0.44}{{#1}}}
    \newcommand{\FloatTok}[1]{\textcolor[rgb]{0.25,0.63,0.44}{{#1}}}
    \newcommand{\CharTok}[1]{\textcolor[rgb]{0.25,0.44,0.63}{{#1}}}
    \newcommand{\StringTok}[1]{\textcolor[rgb]{0.25,0.44,0.63}{{#1}}}
    \newcommand{\CommentTok}[1]{\textcolor[rgb]{0.38,0.63,0.69}{\textit{{#1}}}}
    \newcommand{\OtherTok}[1]{\textcolor[rgb]{0.00,0.44,0.13}{{#1}}}
    \newcommand{\AlertTok}[1]{\textcolor[rgb]{1.00,0.00,0.00}{\textbf{{#1}}}}
    \newcommand{\FunctionTok}[1]{\textcolor[rgb]{0.02,0.16,0.49}{{#1}}}
    \newcommand{\RegionMarkerTok}[1]{{#1}}
    \newcommand{\ErrorTok}[1]{\textcolor[rgb]{1.00,0.00,0.00}{\textbf{{#1}}}}
    \newcommand{\NormalTok}[1]{{#1}}
    
    % Additional commands for more recent versions of Pandoc
    \newcommand{\ConstantTok}[1]{\textcolor[rgb]{0.53,0.00,0.00}{{#1}}}
    \newcommand{\SpecialCharTok}[1]{\textcolor[rgb]{0.25,0.44,0.63}{{#1}}}
    \newcommand{\VerbatimStringTok}[1]{\textcolor[rgb]{0.25,0.44,0.63}{{#1}}}
    \newcommand{\SpecialStringTok}[1]{\textcolor[rgb]{0.73,0.40,0.53}{{#1}}}
    \newcommand{\ImportTok}[1]{{#1}}
    \newcommand{\DocumentationTok}[1]{\textcolor[rgb]{0.73,0.13,0.13}{\textit{{#1}}}}
    \newcommand{\AnnotationTok}[1]{\textcolor[rgb]{0.38,0.63,0.69}{\textbf{\textit{{#1}}}}}
    \newcommand{\CommentVarTok}[1]{\textcolor[rgb]{0.38,0.63,0.69}{\textbf{\textit{{#1}}}}}
    \newcommand{\VariableTok}[1]{\textcolor[rgb]{0.10,0.09,0.49}{{#1}}}
    \newcommand{\ControlFlowTok}[1]{\textcolor[rgb]{0.00,0.44,0.13}{\textbf{{#1}}}}
    \newcommand{\OperatorTok}[1]{\textcolor[rgb]{0.40,0.40,0.40}{{#1}}}
    \newcommand{\BuiltInTok}[1]{{#1}}
    \newcommand{\ExtensionTok}[1]{{#1}}
    \newcommand{\PreprocessorTok}[1]{\textcolor[rgb]{0.74,0.48,0.00}{{#1}}}
    \newcommand{\AttributeTok}[1]{\textcolor[rgb]{0.49,0.56,0.16}{{#1}}}
    \newcommand{\InformationTok}[1]{\textcolor[rgb]{0.38,0.63,0.69}{\textbf{\textit{{#1}}}}}
    \newcommand{\WarningTok}[1]{\textcolor[rgb]{0.38,0.63,0.69}{\textbf{\textit{{#1}}}}}
    
    
    % Define a nice break command that doesn't care if a line doesn't already
    % exist.
    \def\br{\hspace*{\fill} \\* }
    % Math Jax compatability definitions
    \def\gt{>}
    \def\lt{<}
    % Document parameters
    \title{Calculus\_A5}
    
    
    

    % Pygments definitions
    
\makeatletter
\def\PY@reset{\let\PY@it=\relax \let\PY@bf=\relax%
    \let\PY@ul=\relax \let\PY@tc=\relax%
    \let\PY@bc=\relax \let\PY@ff=\relax}
\def\PY@tok#1{\csname PY@tok@#1\endcsname}
\def\PY@toks#1+{\ifx\relax#1\empty\else%
    \PY@tok{#1}\expandafter\PY@toks\fi}
\def\PY@do#1{\PY@bc{\PY@tc{\PY@ul{%
    \PY@it{\PY@bf{\PY@ff{#1}}}}}}}
\def\PY#1#2{\PY@reset\PY@toks#1+\relax+\PY@do{#2}}

\expandafter\def\csname PY@tok@gd\endcsname{\def\PY@tc##1{\textcolor[rgb]{0.63,0.00,0.00}{##1}}}
\expandafter\def\csname PY@tok@gu\endcsname{\let\PY@bf=\textbf\def\PY@tc##1{\textcolor[rgb]{0.50,0.00,0.50}{##1}}}
\expandafter\def\csname PY@tok@gt\endcsname{\def\PY@tc##1{\textcolor[rgb]{0.00,0.27,0.87}{##1}}}
\expandafter\def\csname PY@tok@gs\endcsname{\let\PY@bf=\textbf}
\expandafter\def\csname PY@tok@gr\endcsname{\def\PY@tc##1{\textcolor[rgb]{1.00,0.00,0.00}{##1}}}
\expandafter\def\csname PY@tok@cm\endcsname{\let\PY@it=\textit\def\PY@tc##1{\textcolor[rgb]{0.25,0.50,0.50}{##1}}}
\expandafter\def\csname PY@tok@vg\endcsname{\def\PY@tc##1{\textcolor[rgb]{0.10,0.09,0.49}{##1}}}
\expandafter\def\csname PY@tok@vi\endcsname{\def\PY@tc##1{\textcolor[rgb]{0.10,0.09,0.49}{##1}}}
\expandafter\def\csname PY@tok@vm\endcsname{\def\PY@tc##1{\textcolor[rgb]{0.10,0.09,0.49}{##1}}}
\expandafter\def\csname PY@tok@mh\endcsname{\def\PY@tc##1{\textcolor[rgb]{0.40,0.40,0.40}{##1}}}
\expandafter\def\csname PY@tok@cs\endcsname{\let\PY@it=\textit\def\PY@tc##1{\textcolor[rgb]{0.25,0.50,0.50}{##1}}}
\expandafter\def\csname PY@tok@ge\endcsname{\let\PY@it=\textit}
\expandafter\def\csname PY@tok@vc\endcsname{\def\PY@tc##1{\textcolor[rgb]{0.10,0.09,0.49}{##1}}}
\expandafter\def\csname PY@tok@il\endcsname{\def\PY@tc##1{\textcolor[rgb]{0.40,0.40,0.40}{##1}}}
\expandafter\def\csname PY@tok@go\endcsname{\def\PY@tc##1{\textcolor[rgb]{0.53,0.53,0.53}{##1}}}
\expandafter\def\csname PY@tok@cp\endcsname{\def\PY@tc##1{\textcolor[rgb]{0.74,0.48,0.00}{##1}}}
\expandafter\def\csname PY@tok@gi\endcsname{\def\PY@tc##1{\textcolor[rgb]{0.00,0.63,0.00}{##1}}}
\expandafter\def\csname PY@tok@gh\endcsname{\let\PY@bf=\textbf\def\PY@tc##1{\textcolor[rgb]{0.00,0.00,0.50}{##1}}}
\expandafter\def\csname PY@tok@ni\endcsname{\let\PY@bf=\textbf\def\PY@tc##1{\textcolor[rgb]{0.60,0.60,0.60}{##1}}}
\expandafter\def\csname PY@tok@nl\endcsname{\def\PY@tc##1{\textcolor[rgb]{0.63,0.63,0.00}{##1}}}
\expandafter\def\csname PY@tok@nn\endcsname{\let\PY@bf=\textbf\def\PY@tc##1{\textcolor[rgb]{0.00,0.00,1.00}{##1}}}
\expandafter\def\csname PY@tok@no\endcsname{\def\PY@tc##1{\textcolor[rgb]{0.53,0.00,0.00}{##1}}}
\expandafter\def\csname PY@tok@na\endcsname{\def\PY@tc##1{\textcolor[rgb]{0.49,0.56,0.16}{##1}}}
\expandafter\def\csname PY@tok@nb\endcsname{\def\PY@tc##1{\textcolor[rgb]{0.00,0.50,0.00}{##1}}}
\expandafter\def\csname PY@tok@nc\endcsname{\let\PY@bf=\textbf\def\PY@tc##1{\textcolor[rgb]{0.00,0.00,1.00}{##1}}}
\expandafter\def\csname PY@tok@nd\endcsname{\def\PY@tc##1{\textcolor[rgb]{0.67,0.13,1.00}{##1}}}
\expandafter\def\csname PY@tok@ne\endcsname{\let\PY@bf=\textbf\def\PY@tc##1{\textcolor[rgb]{0.82,0.25,0.23}{##1}}}
\expandafter\def\csname PY@tok@nf\endcsname{\def\PY@tc##1{\textcolor[rgb]{0.00,0.00,1.00}{##1}}}
\expandafter\def\csname PY@tok@si\endcsname{\let\PY@bf=\textbf\def\PY@tc##1{\textcolor[rgb]{0.73,0.40,0.53}{##1}}}
\expandafter\def\csname PY@tok@s2\endcsname{\def\PY@tc##1{\textcolor[rgb]{0.73,0.13,0.13}{##1}}}
\expandafter\def\csname PY@tok@nt\endcsname{\let\PY@bf=\textbf\def\PY@tc##1{\textcolor[rgb]{0.00,0.50,0.00}{##1}}}
\expandafter\def\csname PY@tok@nv\endcsname{\def\PY@tc##1{\textcolor[rgb]{0.10,0.09,0.49}{##1}}}
\expandafter\def\csname PY@tok@s1\endcsname{\def\PY@tc##1{\textcolor[rgb]{0.73,0.13,0.13}{##1}}}
\expandafter\def\csname PY@tok@dl\endcsname{\def\PY@tc##1{\textcolor[rgb]{0.73,0.13,0.13}{##1}}}
\expandafter\def\csname PY@tok@ch\endcsname{\let\PY@it=\textit\def\PY@tc##1{\textcolor[rgb]{0.25,0.50,0.50}{##1}}}
\expandafter\def\csname PY@tok@m\endcsname{\def\PY@tc##1{\textcolor[rgb]{0.40,0.40,0.40}{##1}}}
\expandafter\def\csname PY@tok@gp\endcsname{\let\PY@bf=\textbf\def\PY@tc##1{\textcolor[rgb]{0.00,0.00,0.50}{##1}}}
\expandafter\def\csname PY@tok@sh\endcsname{\def\PY@tc##1{\textcolor[rgb]{0.73,0.13,0.13}{##1}}}
\expandafter\def\csname PY@tok@ow\endcsname{\let\PY@bf=\textbf\def\PY@tc##1{\textcolor[rgb]{0.67,0.13,1.00}{##1}}}
\expandafter\def\csname PY@tok@sx\endcsname{\def\PY@tc##1{\textcolor[rgb]{0.00,0.50,0.00}{##1}}}
\expandafter\def\csname PY@tok@bp\endcsname{\def\PY@tc##1{\textcolor[rgb]{0.00,0.50,0.00}{##1}}}
\expandafter\def\csname PY@tok@c1\endcsname{\let\PY@it=\textit\def\PY@tc##1{\textcolor[rgb]{0.25,0.50,0.50}{##1}}}
\expandafter\def\csname PY@tok@fm\endcsname{\def\PY@tc##1{\textcolor[rgb]{0.00,0.00,1.00}{##1}}}
\expandafter\def\csname PY@tok@o\endcsname{\def\PY@tc##1{\textcolor[rgb]{0.40,0.40,0.40}{##1}}}
\expandafter\def\csname PY@tok@kc\endcsname{\let\PY@bf=\textbf\def\PY@tc##1{\textcolor[rgb]{0.00,0.50,0.00}{##1}}}
\expandafter\def\csname PY@tok@c\endcsname{\let\PY@it=\textit\def\PY@tc##1{\textcolor[rgb]{0.25,0.50,0.50}{##1}}}
\expandafter\def\csname PY@tok@mf\endcsname{\def\PY@tc##1{\textcolor[rgb]{0.40,0.40,0.40}{##1}}}
\expandafter\def\csname PY@tok@err\endcsname{\def\PY@bc##1{\setlength{\fboxsep}{0pt}\fcolorbox[rgb]{1.00,0.00,0.00}{1,1,1}{\strut ##1}}}
\expandafter\def\csname PY@tok@mb\endcsname{\def\PY@tc##1{\textcolor[rgb]{0.40,0.40,0.40}{##1}}}
\expandafter\def\csname PY@tok@ss\endcsname{\def\PY@tc##1{\textcolor[rgb]{0.10,0.09,0.49}{##1}}}
\expandafter\def\csname PY@tok@sr\endcsname{\def\PY@tc##1{\textcolor[rgb]{0.73,0.40,0.53}{##1}}}
\expandafter\def\csname PY@tok@mo\endcsname{\def\PY@tc##1{\textcolor[rgb]{0.40,0.40,0.40}{##1}}}
\expandafter\def\csname PY@tok@kd\endcsname{\let\PY@bf=\textbf\def\PY@tc##1{\textcolor[rgb]{0.00,0.50,0.00}{##1}}}
\expandafter\def\csname PY@tok@mi\endcsname{\def\PY@tc##1{\textcolor[rgb]{0.40,0.40,0.40}{##1}}}
\expandafter\def\csname PY@tok@kn\endcsname{\let\PY@bf=\textbf\def\PY@tc##1{\textcolor[rgb]{0.00,0.50,0.00}{##1}}}
\expandafter\def\csname PY@tok@cpf\endcsname{\let\PY@it=\textit\def\PY@tc##1{\textcolor[rgb]{0.25,0.50,0.50}{##1}}}
\expandafter\def\csname PY@tok@kr\endcsname{\let\PY@bf=\textbf\def\PY@tc##1{\textcolor[rgb]{0.00,0.50,0.00}{##1}}}
\expandafter\def\csname PY@tok@s\endcsname{\def\PY@tc##1{\textcolor[rgb]{0.73,0.13,0.13}{##1}}}
\expandafter\def\csname PY@tok@kp\endcsname{\def\PY@tc##1{\textcolor[rgb]{0.00,0.50,0.00}{##1}}}
\expandafter\def\csname PY@tok@w\endcsname{\def\PY@tc##1{\textcolor[rgb]{0.73,0.73,0.73}{##1}}}
\expandafter\def\csname PY@tok@kt\endcsname{\def\PY@tc##1{\textcolor[rgb]{0.69,0.00,0.25}{##1}}}
\expandafter\def\csname PY@tok@sc\endcsname{\def\PY@tc##1{\textcolor[rgb]{0.73,0.13,0.13}{##1}}}
\expandafter\def\csname PY@tok@sb\endcsname{\def\PY@tc##1{\textcolor[rgb]{0.73,0.13,0.13}{##1}}}
\expandafter\def\csname PY@tok@sa\endcsname{\def\PY@tc##1{\textcolor[rgb]{0.73,0.13,0.13}{##1}}}
\expandafter\def\csname PY@tok@k\endcsname{\let\PY@bf=\textbf\def\PY@tc##1{\textcolor[rgb]{0.00,0.50,0.00}{##1}}}
\expandafter\def\csname PY@tok@se\endcsname{\let\PY@bf=\textbf\def\PY@tc##1{\textcolor[rgb]{0.73,0.40,0.13}{##1}}}
\expandafter\def\csname PY@tok@sd\endcsname{\let\PY@it=\textit\def\PY@tc##1{\textcolor[rgb]{0.73,0.13,0.13}{##1}}}

\def\PYZbs{\char`\\}
\def\PYZus{\char`\_}
\def\PYZob{\char`\{}
\def\PYZcb{\char`\}}
\def\PYZca{\char`\^}
\def\PYZam{\char`\&}
\def\PYZlt{\char`\<}
\def\PYZgt{\char`\>}
\def\PYZsh{\char`\#}
\def\PYZpc{\char`\%}
\def\PYZdl{\char`\$}
\def\PYZhy{\char`\-}
\def\PYZsq{\char`\'}
\def\PYZdq{\char`\"}
\def\PYZti{\char`\~}
% for compatibility with earlier versions
\def\PYZat{@}
\def\PYZlb{[}
\def\PYZrb{]}
\makeatother


    % Exact colors from NB
    \definecolor{incolor}{rgb}{0.0, 0.0, 0.5}
    \definecolor{outcolor}{rgb}{0.545, 0.0, 0.0}



    
    % Prevent overflowing lines due to hard-to-break entities
    \sloppy 
    % Setup hyperref package
    \hypersetup{
      breaklinks=true,  % so long urls are correctly broken across lines
      colorlinks=true,
      urlcolor=urlcolor,
      linkcolor=linkcolor,
      citecolor=citecolor,
      }
    % Slightly bigger margins than the latex defaults
    
    \geometry{verbose,tmargin=1in,bmargin=1in,lmargin=1in,rmargin=1in}
    
    

    \begin{document}
    
    
    \maketitle
    
    

    
    Assignment 5

Michelle S Lee

CS111A Fall 2018

    \section{Part A}\label{part-a}

\subsection{A.1 Handle the heat}\label{a.1-handle-the-heat}

Let \(x\) be the length of east/west walls, \(y\) be the length of
north/ south walls, and \(z\) the height of the building

    \subsection{(a)}\label{a}

\textbf{Setting up the functions and variables}

Constraints \[x >= 30\] \[y >= 30\] \[z >= 4\] \[xyz = 400\]

Objective function (to be minimized): \[f(x,y) = 20xz + 16yz + 6xy\]

Constraint functions: \[x >= 30\] \[y >= 30\] \[z >= 4\] \[xyz = 4000\]

\textbf{simplify by removing a variable from the equation}\\
\(xyz = 4000\)\\
\(z = \frac {4000}{xy}\)

Subbing \(z = 400/xy\) into objective function:\\
\[f(x,y) = 20x\frac {4000}{xy} + 16y\frac {4000}{xy} + 6xy\]
\[= \frac {80000}{y} + \frac {64000}{x} + 6xy\]

subbing \(z = 400/xy\) into constraint function:\\
\(\frac {4000}{xy} >= 4\)\\
\(\frac {1000}{y} >= x\)\\
\(\frac {1000}{x} >= y\)

simplified constraint:\\
\[ 30 <= x <= \frac {1000}{y}\] \[30 <= y <= \frac {1000}{x}\]

    \begin{Verbatim}[commandchars=\\\{\}]
{\color{incolor}In [{\color{incolor}14}]:} \PY{c+c1}{\PYZsh{} plotting constraints on a contour plot }
         \PY{k+kn}{from} \PY{n+nn}{sage.plot.contour\PYZus{}plot} \PY{k+kn}{import} \PY{n}{ContourPlot}
         \PY{k+kn}{import} \PY{n+nn}{matplotlib.cm}\PY{p}{;} \PY{n}{matplotlib}\PY{o}{.}\PY{n}{cm}\PY{o}{.}\PY{n}{datad}\PY{o}{.}\PY{n}{keys}\PY{p}{(}\PY{p}{)}
         \PY{k+kn}{from} \PY{n+nn}{sage.plot.line} \PY{k+kn}{import} \PY{n}{Line}
         \PY{n}{x}\PY{p}{,}\PY{n}{y} \PY{o}{=} \PY{n}{var}\PY{p}{(}\PY{l+s+s1}{\PYZsq{}}\PY{l+s+s1}{x,y}\PY{l+s+s1}{\PYZsq{}}\PY{p}{)}
         \PY{n}{f}\PY{p}{(}\PY{n}{x}\PY{p}{,}\PY{n}{y}\PY{p}{)} \PY{o}{=} \PY{l+m+mi}{80000}\PY{o}{/}\PY{n}{y} \PY{o}{+} \PY{l+m+mi}{64000}\PY{o}{/}\PY{n}{x} \PY{o}{+} \PY{l+m+mi}{6}\PY{o}{*}\PY{n}{x}\PY{o}{*}\PY{n}{y}
         \PY{n}{C} \PY{o}{=} \PY{n}{contour\PYZus{}plot}\PY{p}{(}\PY{n}{f}\PY{p}{,} \PY{p}{(}\PY{l+m+mi}{28}\PY{p}{,}\PY{l+m+mi}{35}\PY{p}{)}\PY{p}{,} \PY{p}{(}\PY{l+m+mi}{25}\PY{p}{,}\PY{l+m+mi}{36}\PY{p}{)}\PY{p}{,}\PY{n}{cmap}\PY{o}{=}\PY{l+s+s1}{\PYZsq{}}\PY{l+s+s1}{Purples}\PY{l+s+s1}{\PYZsq{}}\PY{p}{,} \PY{n}{colorbar}\PY{o}{=}\PY{n+nb+bp}{True}\PY{p}{,} \PY{n}{axes\PYZus{}labels}\PY{o}{=}\PY{p}{[}\PY{l+s+s1}{\PYZsq{}}\PY{l+s+s1}{\PYZdl{}x\PYZdl{}}\PY{l+s+s1}{\PYZsq{}}\PY{p}{,}\PY{l+s+s1}{\PYZsq{}}\PY{l+s+s1}{\PYZdl{}y\PYZdl{}}\PY{l+s+s1}{\PYZsq{}}\PY{p}{]}\PY{p}{)}
         \PY{n}{c1} \PY{o}{=} \PY{n}{line}\PY{p}{(}\PY{p}{[}\PY{p}{(}\PY{l+m+mi}{30}\PY{p}{,}\PY{l+m+mi}{25}\PY{p}{)}\PY{p}{,} \PY{p}{(}\PY{l+m+mi}{30}\PY{p}{,}\PY{l+m+mi}{36}\PY{p}{)}\PY{p}{]}\PY{p}{,}\PY{n}{color}\PY{o}{=}\PY{l+s+s2}{\PYZdq{}}\PY{l+s+s2}{red}\PY{l+s+s2}{\PYZdq{}}\PY{p}{)}
         \PY{n}{c2} \PY{o}{=} \PY{n}{plot}\PY{p}{(}\PY{l+m+mi}{1000}\PY{o}{/}\PY{n}{x}\PY{p}{,}\PY{p}{(}\PY{n}{x}\PY{p}{,}\PY{l+m+mi}{28}\PY{p}{,}\PY{l+m+mi}{35}\PY{p}{)}\PY{p}{,}\PY{n}{color}\PY{o}{=}\PY{l+s+s2}{\PYZdq{}}\PY{l+s+s2}{blue}\PY{l+s+s2}{\PYZdq{}}\PY{p}{)}
         \PY{n}{c3} \PY{o}{=} \PY{n}{plot}\PY{p}{(}\PY{l+m+mi}{30}\PY{p}{,}\PY{p}{(}\PY{n}{x}\PY{p}{,}\PY{l+m+mi}{28}\PY{p}{,}\PY{l+m+mi}{35}\PY{p}{)}\PY{p}{,}\PY{n}{color}\PY{o}{=}\PY{l+s+s2}{\PYZdq{}}\PY{l+s+s2}{green}\PY{l+s+s2}{\PYZdq{}}\PY{p}{)}
         \PY{n}{C}\PY{o}{+}\PY{n}{c1}\PY{o}{+}\PY{n}{c2}\PY{o}{+}\PY{n}{c3}
\end{Verbatim}

\texttt{\color{outcolor}Out[{\color{outcolor}14}]:}
    
    \begin{center}
    \adjustimage{max size={0.9\linewidth}{0.9\paperheight}}{output_3_0.png}
    \end{center}
    { \hspace*{\fill} \\}
    

    \begin{Verbatim}[commandchars=\\\{\}]
{\color{incolor}In [{\color{incolor}44}]:} \PY{c+c1}{\PYZsh{} plotting constraints with shading}
         
         \PY{k+kn}{import} \PY{n+nn}{numpy} \PY{k+kn}{as} \PY{n+nn}{np}
         \PY{k+kn}{import} \PY{n+nn}{matplotlib.pyplot} \PY{k+kn}{as} \PY{n+nn}{plt}
         
         \PY{n}{x1} \PY{o}{=} \PY{n}{np}\PY{o}{.}\PY{n}{linspace}\PY{p}{(}\PY{l+m+mi}{0}\PY{p}{,}\PY{l+m+mi}{40}\PY{p}{,}\PY{l+m+mi}{40}\PY{p}{)}
         \PY{n}{y1} \PY{o}{=} \PY{n}{np}\PY{o}{.}\PY{n}{full}\PY{p}{(}\PY{l+m+mi}{40}\PY{p}{,}\PY{l+m+mi}{30}\PY{p}{)}
         
         \PY{n}{x2} \PY{o}{=} \PY{n}{np}\PY{o}{.}\PY{n}{full}\PY{p}{(}\PY{l+m+mi}{40}\PY{p}{,}\PY{l+m+mi}{30}\PY{p}{)}
         \PY{n}{y2} \PY{o}{=} \PY{n}{np}\PY{o}{.}\PY{n}{linspace}\PY{p}{(}\PY{l+m+mi}{0}\PY{p}{,}\PY{l+m+mi}{40}\PY{p}{,}\PY{l+m+mi}{40}\PY{p}{)}
         \PY{n}{plt}\PY{o}{.}\PY{n}{plot}\PY{p}{(}\PY{n}{x1}\PY{p}{,}\PY{n}{y1}\PY{p}{)}
         \PY{n}{plt}\PY{o}{.}\PY{n}{plot}\PY{p}{(}\PY{n}{x2}\PY{p}{,}\PY{n}{y2}\PY{p}{)}
         \PY{n}{plt}\PY{o}{.}\PY{n}{plot}\PY{p}{(}\PY{n}{x1}\PY{p}{,}\PY{l+m+mi}{1000}\PY{o}{/}\PY{n}{x1}\PY{p}{)}
         \PY{n}{plt}\PY{o}{.}\PY{n}{ylim}\PY{p}{(}\PY{l+m+mi}{0}\PY{p}{,} \PY{l+m+mi}{40}\PY{p}{)}
         \PY{n}{plt}\PY{o}{.}\PY{n}{xlim}\PY{p}{(}\PY{l+m+mi}{0}\PY{p}{,} \PY{l+m+mi}{40}\PY{p}{)}
         \PY{n}{plt}\PY{o}{.}\PY{n}{fill\PYZus{}between}\PY{p}{(}\PY{n}{x1}\PY{p}{,} \PY{l+m+mi}{0}\PY{p}{,} \PY{n}{y1}\PY{p}{,} \PY{n}{facecolor}\PY{o}{=}\PY{l+s+s1}{\PYZsq{}}\PY{l+s+s1}{blue}\PY{l+s+s1}{\PYZsq{}}\PY{p}{,} \PY{n}{alpha}\PY{o}{=}\PY{l+m+mf}{0.3}\PY{p}{)}
         \PY{n}{plt}\PY{o}{.}\PY{n}{fill\PYZus{}between}\PY{p}{(}\PY{n}{x1}\PY{p}{,} \PY{l+m+mi}{0}\PY{p}{,} \PY{l+m+mi}{40}\PY{p}{,} \PY{n}{where}\PY{o}{=}\PY{n}{x1}\PY{o}{\PYZlt{}}\PY{o}{=}\PY{l+m+mi}{30}\PY{p}{,}\PY{n}{facecolor}\PY{o}{=}\PY{l+s+s1}{\PYZsq{}}\PY{l+s+s1}{yellow}\PY{l+s+s1}{\PYZsq{}}\PY{p}{,} \PY{n}{alpha}\PY{o}{=}\PY{l+m+mf}{0.3}\PY{p}{)}
         \PY{n}{plt}\PY{o}{.}\PY{n}{fill\PYZus{}between}\PY{p}{(}\PY{n}{x1}\PY{p}{,} \PY{l+m+mi}{1000}\PY{o}{/}\PY{n}{x1}\PY{p}{,}\PY{l+m+mi}{40}\PY{p}{,}\PY{n}{facecolor}\PY{o}{=}\PY{l+s+s1}{\PYZsq{}}\PY{l+s+s1}{green}\PY{l+s+s1}{\PYZsq{}}\PY{p}{,} \PY{n}{alpha}\PY{o}{=}\PY{l+m+mf}{0.3}\PY{p}{)}
         \PY{n}{plt}\PY{o}{.}\PY{n}{title}\PY{p}{(}\PY{l+s+s2}{\PYZdq{}}\PY{l+s+s2}{Constraints on Input Variables}\PY{l+s+s2}{\PYZdq{}}\PY{p}{)}
         \PY{n}{plt}\PY{o}{.}\PY{n}{xlabel}\PY{p}{(}\PY{l+s+s2}{\PYZdq{}}\PY{l+s+s2}{Length of East/West walls}\PY{l+s+s2}{\PYZdq{}}\PY{p}{)}
         \PY{n}{plt}\PY{o}{.}\PY{n}{ylabel}\PY{p}{(}\PY{l+s+s2}{\PYZdq{}}\PY{l+s+s2}{Length of North/South walls}\PY{l+s+s2}{\PYZdq{}}\PY{p}{)}
\end{Verbatim}


    \begin{Verbatim}[commandchars=\\\{\}]
/opt/sagemath-8.3/local/lib/python2.7/site-packages/sage/repl/ipython\_kernel/\_\_main\_\_.py:13: RuntimeWarning: divide by zero encountered in divide
/opt/sagemath-8.3/local/lib/python2.7/site-packages/sage/repl/ipython\_kernel/\_\_main\_\_.py:18: RuntimeWarning: divide by zero encountered in divide

    \end{Verbatim}

\begin{Verbatim}[commandchars=\\\{\}]
{\color{outcolor}Out[{\color{outcolor}44}]:} Text(0,0.5,u'Length of North/South walls')
\end{Verbatim}
            
    \begin{center}
    \adjustimage{max size={0.9\linewidth}{0.9\paperheight}}{output_4_2.png}
    \end{center}
    { \hspace*{\fill} \\}
    
    \subsection{(b)}\label{b}

    \textbf{Find Critical Points of Objective Function}\\
Critical points are when the rate of change in any direction = 0

Rate of change on the x axis:\\
\(\frac{\delta f}{\delta x} = -64000x^-2 + 6y\)\\
\emph{sub in rate of change = 0}\\
\(6y = 64000x^-2\)\\
\(y = \frac {32000}{3x^2}\)

Rate of change on the y axis:\\
\(\frac{\delta f}{\delta y} = -80000y^-2 + 6x\)\\
\emph{sub in rate of change = 0}\\
\(0 = -80000y^-2 + 6x\)\\
\emph{sub in \(y = -\frac {32000}{3x^2}\)}\\
\(0 = -80000(-\frac {32000}{3x^2})^-2 + 6x\)\\
\(0 = -80000(-\frac {3x^2}{32000})^2 + 6x\)\\
\(6x = 80000(\frac {9x^4}{1024000000})\)\\
\(6x = (\frac {9x^4}{12800})\)\\
\(x^3 = \frac{12800*6}{9}\)\\
\(x =(\frac{25600}{3})^\frac{1}{3}\)\\
\(x \approx 20.435\)

Sub \(x \approx 20.435\) to find y:\\
\(y = \frac {32000}{3(((\frac{25600}{3})^\frac{1}{3})^2}\)\\
\(y = \frac {32000}{3(\frac{25600}{3})^\frac{2}{3}}\)\\
\(y \approx 25.544\)

Critical Point of objective function:\\
\[(20.435,25.544)\]\\
Citical point is not within constraints\\
Thus we have to check the boundaries

    \begin{center}\rule{0.5\linewidth}{\linethickness}\end{center}

\textbf{Checking Boundary 1: \(x = 30\)}\\
Endpoints of boundary 1 (\(y\) values): \([30,\frac{1000}{x}]\)\\
Endpoints of boundary 1 (\(y\) values): \([30,\frac{1000}{30}]\)

\[f(30,y) = \frac {80000}{y} + \frac {64000}{30} + 6(30)y\]\\
\[f(30,y) = \frac {80000}{y} + \frac {6400}{3} + 180y\]

\(\frac{\delta f}{\delta y} = 180-\frac {80000}{y^2}\)\\
Critical Points for along \(x = 30\):\\
\(0 = 180-\frac {80000}{y^2}\)\\
\(\frac {80000}{y^2} = 180\)\\
\(\frac {80000}{180} = y^2\)\\
\(y = (\frac {80000}{180})^\frac {1}{2}\)\\
\(y \approx 21.082\)

Nature of Critical Point along \(x = 30\):\\
\(\frac{\delta f^2}{\delta y^2} = 2\frac {80000}{y^3}\)\\
Sub in \(y \approx 21.082\)\\
\(\frac{\delta f^2}{\delta y^2} = 2\frac {80000}{21.082^3}\)\\
\(\frac{\delta f^2}{\delta y^2} \approx 17.075\) at
\(y \approx 21.082\)\\
\(\frac{\delta f^2}{\delta y^2}\) at \(y \approx 21.082\) is positive,
indicating a minimum point.

Along the line \(x = 30\), there is only a minimum point at
\(y \approx 21.082\). Since the minimmum point lies before the start
point of boundary 1 and we know that there are no other critical points
along \(x = 30\), we know that \(f(30,y)\) is an increasing function
within the constraints of our boundary.\\
Hence the minimum on this boundary is at \(y=30\) and maximum is at
\(y=\frac {1000}{x} = \frac {1000}{30}\)\\
\textbf{Minimum:}\\
\[f(30,30) = \frac {80000}{30} + \frac {64000}{30} + 6(30)(30)\]\\
\[f(30,30) \approx 10200\]\\
\textbf{Maximum:}\\
\[f(30,\frac {1000}{30}) = \frac {80000}{\frac {1000}{30}} + \frac {64000}{30} + 6(30)(\frac {1000}{30})\]\\
\[f(30,\frac {1000}{30}) \approx 10533\]\\
\_\_\_\_\_\_\_\_

    \begin{center}\rule{0.5\linewidth}{\linethickness}\end{center}

\textbf{Checking Boundary 2: \(y = 30\)}\\
Endpoints of boundary 2 (\(x\) values): \([30,\frac{1000}{y}]\)\\
Endpoints of boundary 2 (\(x\) values): \([30,\frac{1000}{30}]\)

\[f(x,30) = \frac {80000}{30} + \frac {64000}{x} + 6x(30)\]\\
\[f(x,30) = \frac {80000}{30} + \frac {64000}{x} + 180x\]

\(\frac{\delta f}{\delta x} = 180-\frac {64000}{x^2}\)\\
Critical Points for Boundary 2:\\
\(0 = 180-\frac {64000}{x^2}\)\\
\(\frac {64000}{x^2} = 180\)\\
\(\frac {64000}{180} = x^2\)\\
\(x = (\frac {64000}{180})^\frac{1}{2}\)\\
\(x \approx 18.856\)

Nature of Critical Point on Boundary 2:\\
\(\frac{\delta f^2}{\delta x^2} = 2\frac {64000}{x^3}\)\\
\(\frac{\delta f^2}{\delta x^2} \approx 0.210\) at
\(x \approx 18.856\)\\
\(\frac{\delta f^2}{\delta x^2}\) is positive, indicating a minimum
point.

Along the line \(y = 30\) , there is only a minimum point at
\(x \approx 18.856\). Since the minimmum point lies before the start
point of boundary 2 and we know that there are no other critical points
along \(y = 30\), we know that \(f(x,30)\) is an increasing function
within the constraints of our boundary.

Hence the minimum on this boundary is at \(x=30\) and maximum is at
\(x=\frac {1000}{y} = \frac {1000}{30}\)

\textbf{Minimum:}\\
\[f(30,30) = \frac {80000}{30} + \frac {64000}{30} + 6(30)(30)\]\\
\[f(30,30) \approx 10200\]\\
\textbf{Maximum:}\\
\[f(30,\frac {1000}{30}) = \frac {80000}{\frac {1000}{30}} + \frac {64000}{30} + 6(30)(\frac {1000}{30})\]\\
\[f(30,\frac {1000}{30}) \approx 10533\]\\
\_\_\_\_\_\_\_\_

    \begin{center}\rule{0.5\linewidth}{\linethickness}\end{center}

\textbf{Checking Boundary 3: \(x = \frac{1000}{y}\)}\\
Endpoints of boundary 3 (\(y\) values): \([30,\frac{1000}{x}]\)\\
Endpoints of boundary 3 (\(y\) values): \([30,\frac{1000}{30}]\)

\[f(\frac{1000}{y},y) = \frac {80000}{y} + \frac {64000}{\frac{1000}{y}} + 6(\frac{1000}{y})y\]\\
\[f(\frac{1000}{y},y) = \frac {80000}{y} + 64y + 6000\]

\(\frac{\delta f}{\delta y} = 64-\frac{80000}{y^2}\)\\
Critical Points for Boundary 3:\\
\(0 = 64-\frac{80000}{y^2}\)\\
\(\frac{80000}{y^2} = 64\)\\
\(\frac{80000}{64} = y^2\)\\
\(y = (\frac{80000}{64})^\frac{1}{2}\)\\
\(y \approx 35.355\)

Nature of Critical Point on Boundary 3:\\
\(\frac{\delta f^2}{\delta y^2} = 2\frac {80000}{y^3}\)\\
\(\frac{\delta f^2}{\delta x^2} \approx 1.105\) at
\(y \approx 35.355\)\\
\(\frac{\delta f^2}{\delta x^2}\) is positive, indicating a minimum
point.

Along the line \(x = \frac{1000}{y}\), there is only a minimum point at
\(x \approx 18.856\). Since the minimmum point lies after the ending
point of boundary 3 and we know that there are no other critical points
along \(x = \frac{1000}{y}\), we know that \(f(\frac{1000}{y},y)\) is an
increasing function within the constraints of our boundary.

Hence the minimum on this boundary is at \(x=30\) and maximum is at
\(x=\frac {1000}{y} = \frac {1000}{30}\)

\textbf{Minimum:}\\
\[f(30,30) = \frac {80000}{30} + \frac {64000}{30} + 6(30)(30)\]\\
\[f(30,30) \approx 10200\]\\
\textbf{Maximum:}\\
\[f(30,\frac {1000}{30}) = \frac {80000}{\frac {1000}{30}} + \frac {64000}{30} + 6(30)(\frac {1000}{30})\]\\
\[f(30,\frac {1000}{30}) \approx 10533\]\\
\_\_\_\_\_\_\_\_

\(\frac{\delta f}{\delta x}\)is negative for any value between 30 to
\(\frac{1000}{30}\). Thus \(f(\frac{1000}{y},y)\) is a decreasing
function within the constraint.

Hence the minimum on this boundary is at \(y=\frac {1000}{30}\) and
maximum is at \(y=30\)\\
\emph{Minimum: }\\
\(f(\frac{1000}{\frac {1000}{30}},\frac {1000}{30}) = \frac {80000}{\frac {1000}{30}} + 64\frac {1000}{30} + 6000\)\\
\(f(\frac{1000}{\frac {1000}{30}},\frac {1000}{30}) \approx 10533\)\\
\emph{Maximum: }\\
\(f(\frac{1000}{30},30) = \frac {80000}{30} + 64(30) + 6000\)\\
\(f(\frac{1000}{30},30) \approx 10587\)

    The absolute minimum of this function is 10200 \(\frac {units}{m^2}\) of
heat loss.\\
This is achieved when\\
\[x = 30\]\\
\[y=30\]\\
\[z = \frac {4000}{xy} = \frac {40}{9}\]

    \subsection{(c)}\label{c}

Yes. in question (a) we found that the minimum point on the objective
function (without any constraints) is when \[(x,y) = (20.435,25.544)\]

At this minimum point, we would have the following variable values:\\
\[x = 20.435\]\\
\[y = 25.544\]\\
\[z = \frac {4000}{(20.435)(25.544)} \approx 7.663\]

The heatloss with these variables would be:\\
\[f(20.435,25.544) = 20(20.435)(\frac {4000}{(20.435)(25.544)}) + 16(25.544)(\frac {4000}{(20.435)(25.544)}) + 6(20.435)(25.544)\]
\[\approx 9395.682\]

    \subsection{A.2 Handle the heat}\label{a.2-handle-the-heat}

Objective function (to be maximized):\\
\[f(x,y) = 2x+3y\]

Constraint function:\\
\[c(x,y) = \sqrt{x} + \sqrt{y} = 5\]

    \subsection{(a)}\label{a}

Differentiate Objective function by all variables\\
\(\frac{\delta f}{\delta x} = 2\)\\
\(\frac{\delta f}{\delta y} = 3\)

Differentiate Constraint function by all variables\\
\(\frac{\delta c}{\delta x} = \frac{1}{2\sqrt{x}}\)\\
\(\frac{\delta c}{\delta y} = \frac{1}{2\sqrt{y}}\)

Find the set of equations for stationary points along those variables:\\
Along the x axis:\\
\(\frac{\delta f}{\delta x} = \lambda \frac{\delta c}{\delta x}\)\\
\(2 = \lambda \frac{1}{2\sqrt{x}}\)\\
\(2 = \frac{\lambda}{2\sqrt{x}}\)\\
Along the y axis:\\
\(\frac{\delta f}{\delta y} = \lambda \frac{\delta c}{\delta y}\)\\
\(3 = \lambda \frac{1}{2\sqrt{y}}\)\\
\(3 = \frac{\lambda}{2\sqrt{y}}\)

Find the set(s) of variables that are stationary points within the
constraints:\\
\(2 = \lambda \frac{1}{2\sqrt{x}}\)\\
\(3 = \lambda \frac{1}{2\sqrt{y}}\)\\
\(\sqrt{x} + \sqrt{y} = 5\)

Eq1: \(2 = \lambda \frac{1}{2\sqrt{x}}\)\\
Eq2: \(3 = \lambda \frac{1}{2\sqrt{y}}\)\\
Eq3: \(\sqrt{x} + \sqrt{y} = 5\)

Rearrage Eq 3 to define one variable by the other\\
\(\sqrt{x} + \sqrt{y} = 5\)\\
\(\sqrt{x} = 5 - \sqrt{y}\)\\
\(x = (5 - \sqrt{y})^2\)\\
\(x = 25 - 10\sqrt{y} + y\)

Sub \(x = 25 - 10\sqrt{y} + y\) into Eq1\\
\(2 = \frac{\lambda}{2\sqrt{25 - 10\sqrt{y} + y}}\)\\
\(\lambda= 2*2\sqrt{25 - 10\sqrt{y} + y}\)\\
\(\lambda= 4\sqrt{25 - 10\sqrt{y} + y}\)

Rearrange Eq 2 to define it by \(\lambda\)\\
\(3 = \frac{\lambda}{2\sqrt{y}}\)\\
\(\lambda = 6\sqrt{y}\)

Equate the two equations we have for \(\lambda\) (Eq1 and Eq2)\\
\(6\sqrt{y} = 4\sqrt{25 - 10\sqrt{y} + y}\)

    \(36y= 400-160\sqrt{y}+16y\)\\
\(160\sqrt{y} = 400+16y-36y\)\\
\(160\sqrt{y} = 400-20y\)\\
\(25600y = 160000-16000y + 400y^2\)\\
\(400y^2 - 41600y + 160000 = 0\)

Use the Roots of Quadratic Equations formula\\
\(y = \frac {- (-41600)+\sqrt{(-41600)^2-4*400*160000}}{2*400}\)\\
\(y = \frac {41600+\sqrt{41600^2-256000000}}{800}\)\\
\(y = 100\)\\
Check if the root is real by subbing \(y = 100\) into the original
function \(6\sqrt{100} = 4\sqrt{25 - 10\sqrt{100} + 100}\)\\
\(60 = 20\) which is False.\\
Hence this is not a real root.

\(y = \frac {- (-41600)-\sqrt{(-41600)^2-4*400*160000}}{2*400}\)\\
\(y = \frac {41600-\sqrt{41600^2-256000000}}{800}\)\\
\(y = 4\)\\
Check if the root is real by subbing \(y = 100\) into the original
function.\\
\(6\sqrt{4} = 4\sqrt{25 - 10\sqrt{4} + 4}\)\\
\(12 = 12\) which is True.\\
Hence this is a real root.

Thus the only real root is \(y = 4\)

Sub \(y = 4\) into Eq3 to find \(x\)\\
\(\sqrt{x} + \sqrt{4} = 5\)\\
\(x = (5 - 2)^2\)\\
\(x = 9\)

Thus the maximum value of \(f(x,y) is f(9,4)\)\\
\(f(9,4) = 2(9) + 3(4)\)\\
\(f(9,4) = 30\)

    \subsection{(b)}\label{b}

\(f(25,0) = 2(25) + 3(0)\)\\
\(f(25,0) = 50\)

\(f(25,0)\) does give us a larger value than the "maximum point" we
found in part (a), meaning that Lagrange method of finding maximum point
failed to work.

    \subsection{(c)}\label{c}

Rearrange the constraint function to plot it.\\
\(\sqrt{x} + \sqrt{y} = 5\)

\(y = (5-\sqrt{x})^2\)

    \begin{Verbatim}[commandchars=\\\{\}]
{\color{incolor}In [{\color{incolor}38}]:} \PY{c+c1}{\PYZsh{} plotting constraints on a contour plot }
         \PY{k+kn}{from} \PY{n+nn}{sage.plot.contour\PYZus{}plot} \PY{k+kn}{import} \PY{n}{ContourPlot}
         \PY{k+kn}{import} \PY{n+nn}{matplotlib.cm}\PY{p}{;} \PY{n}{matplotlib}\PY{o}{.}\PY{n}{cm}\PY{o}{.}\PY{n}{datad}\PY{o}{.}\PY{n}{keys}\PY{p}{(}\PY{p}{)}
         \PY{k+kn}{from} \PY{n+nn}{sage.plot.line} \PY{k+kn}{import} \PY{n}{Line}
         \PY{n}{x}\PY{p}{,}\PY{n}{y} \PY{o}{=} \PY{n}{var}\PY{p}{(}\PY{l+s+s1}{\PYZsq{}}\PY{l+s+s1}{x,y}\PY{l+s+s1}{\PYZsq{}}\PY{p}{)}
         \PY{n}{f}\PY{p}{(}\PY{n}{x}\PY{p}{,}\PY{n}{y}\PY{p}{)} \PY{o}{=}  \PY{p}{(}\PY{l+m+mi}{2}\PY{o}{*}\PY{n}{x}\PY{p}{)}\PY{o}{+}\PY{p}{(}\PY{l+m+mi}{3}\PY{o}{*}\PY{n}{y}\PY{p}{)}
         \PY{n}{C} \PY{o}{=} \PY{n}{contour\PYZus{}plot}\PY{p}{(}\PY{n}{f}\PY{p}{,} \PY{p}{(}\PY{l+m+mi}{0}\PY{p}{,}\PY{l+m+mi}{30}\PY{p}{)}\PY{p}{,} \PY{p}{(}\PY{l+m+mi}{0}\PY{p}{,}\PY{l+m+mi}{30}\PY{p}{)}\PY{p}{,}\PY{n}{cmap}\PY{o}{=}\PY{l+s+s1}{\PYZsq{}}\PY{l+s+s1}{Purples}\PY{l+s+s1}{\PYZsq{}}\PY{p}{,} \PY{n}{colorbar}\PY{o}{=}\PY{n+nb+bp}{True}\PY{p}{,} \PY{n}{axes\PYZus{}labels}\PY{o}{=}\PY{p}{[}\PY{l+s+s1}{\PYZsq{}}\PY{l+s+s1}{\PYZdl{}x\PYZdl{}}\PY{l+s+s1}{\PYZsq{}}\PY{p}{,}\PY{l+s+s1}{\PYZsq{}}\PY{l+s+s1}{\PYZdl{}y\PYZdl{}}\PY{l+s+s1}{\PYZsq{}}\PY{p}{]}\PY{p}{)}
         \PY{n}{F} \PY{o}{=} \PY{n}{plot}\PY{p}{(}\PY{p}{(}\PY{l+m+mi}{5}\PY{o}{\PYZhy{}}\PY{n}{sqrt}\PY{p}{(}\PY{n}{x}\PY{p}{)}\PY{p}{)}\PY{o}{\PYZca{}}\PY{l+m+mi}{2}\PY{p}{,}\PY{p}{(}\PY{n}{x}\PY{p}{,}\PY{l+m+mi}{0}\PY{p}{,}\PY{l+m+mi}{30}\PY{p}{)}\PY{p}{,}\PY{n}{color}\PY{o}{=}\PY{l+s+s2}{\PYZdq{}}\PY{l+s+s2}{green}\PY{l+s+s2}{\PYZdq{}}\PY{p}{)}
         \PY{n}{P}\PY{o}{=}\PY{n}{point}\PY{p}{(}\PY{p}{(}\PY{l+m+mi}{9}\PY{p}{,}\PY{l+m+mi}{4}\PY{p}{)}\PY{p}{,} \PY{n}{color}\PY{o}{=}\PY{l+s+s1}{\PYZsq{}}\PY{l+s+s1}{red}\PY{l+s+s1}{\PYZsq{}}\PY{p}{,} \PY{n}{markeredgecolor}\PY{o}{=}\PY{l+s+s1}{\PYZsq{}}\PY{l+s+s1}{red}\PY{l+s+s1}{\PYZsq{}}\PY{p}{,} \PY{n}{size}\PY{o}{=}\PY{l+m+mi}{30}\PY{p}{,}\PY{n}{zorder}\PY{o}{=}\PY{l+m+mi}{4}\PY{p}{)}
         \PY{n}{C}\PY{o}{+}\PY{n}{F}\PY{o}{+}\PY{n}{P}
\end{Verbatim}

\texttt{\color{outcolor}Out[{\color{outcolor}38}]:}
    
    \begin{center}
    \adjustimage{max size={0.9\linewidth}{0.9\paperheight}}{output_17_0.png}
    \end{center}
    { \hspace*{\fill} \\}
    

    Lagrange fails to solve the problem because the actual minimum of the
function occurs at \((25,0)\).

The point \(f(9,4)\) is actually a minimum point, not a maximum point.
Lagrange multipliers method gives us all the critical points (when the
rate of change of the objective function along the constraint path is 0)
within a range but since the function within this range only has one
critical point which is a minimum point, we cannot find the maximum
using lagrange. We instead have to separately check the endpoints.

Additionally at \((25,0)\), the partial derivative of the constraint
function by \(y\) (\(\frac{\delta c}{\delta x}\)) does not exist since
division by 0 is undefined. Thus, the point \((25,0)\) is not in the
domain of the functions on which we apply Lagrange multipliers,
excluding that point from consideration when evaluating maximum points.

    \section{Part B}\label{part-b}

\subsection{B.1}\label{b.1}

Let \(x\) be aluminum, \(y\) be iron, and \(z\) be magnesium.

Objective function (to be minimized):\\
\[f(x,y,z) = 6x+4y+8z\]

Constraint function:\\
\[C(x,y,z) = xyz = 1000\]

    Derive the objective function\\
\(\frac{\delta f}{\delta x} = 6\)\\
\(\frac{\delta f}{\delta y} = 4\)\\
\(\frac{\delta f}{\delta z} = 8\)

Derive the constraint function\\
\(\frac{\delta f}{\delta x} = yz\)\\
\(\frac{\delta f}{\delta y} = xz\)\\
\(\frac{\delta f}{\delta z} = xy\)

Find the set of equations for stationary points along all of those
variables\\
\(6 = \lambda yz\)\\
\(4 = \lambda xz\)\\
\(8 = \lambda xy\)

Solve for the set(s) of variables where those there is stationary points
along all of those variables and is within the constraints\\
Eq1: \(6 = \lambda yz\)\\
Eq2: \(4 = \lambda xz\)\\
Eq3: \(8 = \lambda xy\)\\
Eq4: \(xyz = 1000\)

    \textbf{Redefine all variables by x}\\
Rearrange Eq1 and Eq 2 to make them equate \(\lambda\)\\
\(\lambda = \frac{6}{yz}\)\\
\(\lambda = \frac{4}{xz}\)\\
Equate the two \(\lambda\) functions\\
\(\frac{6}{yz} = \frac{4}{xz}\)\\
\(4yz = 6xz\)\\
\(y = \frac{6x}{4}\)

Rearrange Eq1 and Eq 3 to make them equate \(\lambda\)\\
\(\lambda = \frac{6}{yz}\)\\
\(\lambda = \frac{8}{xy}\)\\
Equate the two \(\lambda\) functions\\
\(\frac{6}{yz} = \frac{8}{xy}\)\\
\(6xy = 8yz\)\\
\(z = \frac{6x}{8}\)

\textbf{Solve for x by subbing all variables with their definitions by x
into Eq4}\\
\(x(\frac{6x}{4})(\frac{6x}{8}) = 1000\)\\
\(\frac{36x^3}{32} = 1000\)\\
\(x = \frac{8000}{9}^\frac{1}{3}\)\\
\(x \approx 9.615\)

\textbf{Solve for the other variables using their definitions by x}\\
\(y = \frac{6(\frac{8000}{9}^\frac{1}{3})}{4}\)\\
\(y \approx 14.422\)

\(z = \frac{6(\frac{8000}{9}^\frac{1}{3})}{8}\)\\
\(z \approx 7.211\)

\textbf{Cost of producing 1000 widgets}\\
\(f(9.615,14.422,7.211) = 6(9.615)+4(14.422)+8(7.211) = 173.066\)

\textbf{Checking the nature of the minimum point}\\
\((1000,1,1)\) satisfies the constraint\\
\[f(1000,1,1) = 6(1000)+4(1)+8(1) = 6012\]

Since Lagrange Multipliers Method always returns all critical points on
the function within the constraints, the method only returned one
critical point, we can deduce that the critical point is a minimum point
since there exists another point on the function within the constraints
that has a cost that is higher than the cost of that point.

Tons of aluminium, iron, and magnesium that would produce 1000 widgets
at the lowest cost:\\
\[x \approx 9.615\]\\
\[y \approx 14.422\]\\
\[z \approx 7.211\]

Cost: \[f(9.615,14.422,7.211) = 173.066\]

    \section{Appendix}\label{appendix}

\textbf{\#deduction:} For all qualitative proof, I wrote explicit
deductive steps to derive the proof from axioms of the mathematical
theorems.\\
\textbf{\#variables:} I explicitly stated the variable assignments and
how I manipulated them to answer the questions.\\
\textbf{\#decision seletcion:} In B1 I used external representation of
the problem by plotting the obstacle and constraint functions as well as
the points that we were interested in. In doing so, I was very quickly
and intuitively able to understand the general overview of what was
going on with the function and the path.


    % Add a bibliography block to the postdoc
    
    
    
    \end{document}
